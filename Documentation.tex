\documentclass{article}
\usepackage{listings}
\usepackage{geometry}
\geometry{a4paper, margin=1in}
\title{StandardModel-Package: Estructura y Documentación}
\begin{document}
\maketitle
\section{Estructura del Paquete}

\begin{verbatim}
StandardModel/
├── StandardModel/
│   ├── __init__.py
│   ├── Graphic.py
│       ├── __init__.py
│       ├── Interfaz.py
│       ├── BG.png
│       ├── UNAM.png
│       ├── Ciencias.png
│   ├── Particles.py
│       ├── __init__.py
│       ├── Bosons.py
│       ├── Leptons.py
│       ├── Quearks.png
│   ├── Systems.py
│       ├── __init__.py
│       ├── Operations.py
│   ├── Vectors.py
│       ├── __init__.py
│       ├── four_vector.py
│       ├── rotation.py
│       ├── three_vector.png
├── Demo.ipynb
├── README.md
├── setup.py
├── LICENSE
├── .gitignore
└── requirements.txt
\end{verbatim}

\section{Descripción de Archivos}
\begin{itemize}
    \item \texttt{StandardModel/\_\_init\_\_.py}: Inicializa el paquete \texttt{StandardModel}.
    \item \texttt{StandardModel/Graphic.py}: Contiene la primer interfaz gráfica de presentación del paquete. \texttt{StandardModel/Particles.py}: Comprende información elemental de las propieades de las partículas del modelo estándar. \texttt{StandardModel/Systems.py} y \texttt{StandardModel/Vectors.py}: Corresponde a funciones útiles en el contexto físico-matemático del modelo. 
    \item \texttt{Demo.ipynb}: Directorio que contiene los archivos de prueba para cada módulo.
    \item \texttt{README.md}:Notebook con pruebas de cada modulo.
    \item \texttt{setup.py}: Script de configuración.
    \item \texttt{LICENSE}: Documento de licencia del paquete.
    \item \texttt{.gitignore}: Archivos y carpetas que Git debería ignorar.
    \item \texttt{requirements.txt}: Lista de dependencias del paquete.
\end{itemize}

\section{Instalación y Ejecución}
Para instalar este paquete de manera local es necesario acceder al directorio correspondiente al paquete:
\begin{verbatim}
cd /../../StandardModel
\end{verbatim}
y una vez ahí ejecutar el comando de instalación en modo editable 
\begin{verbatim}
pip install -e .
\end{verbatim}
Para hacer uso de él importa los módulos o propiedades que desees \\

Ej.
\begin{verbatim}
from StandardModel.Particles.Quarks import *

Particula1 = quark_up
Particula1.description()
\end{verbatim}

\section{Uso del Paquete por un Usuario Externo}
Un usuario puede instalar el paquete ejecutando:

\begin{verbatim}
pip install git+https://github.com/usuario/StandardModel.git
\end{verbatim}

Luego puede usarlo como:

\begin{verbatim}
from First-Package import module1
module1.example_function()
\end{verbatim}

\end{document}
